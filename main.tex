\documentclass{article}
\usepackage[utf8]{inputenc}
\usepackage[spanish]{babel}
\usepackage{listings}
\usepackage{graphicx}
\graphicspath{ {images/} }
\usepackage{cite}

\begin{document}

\begin{titlepage}
    \begin{center}
        \vspace*{1cm}
            
        \Huge
        \textbf{Calistenia}
            
        \vspace{0.5cm}
        \LARGE
        Actividad secuencia de instrucciones
            
        \vspace{1.5cm}
            
        \textbf{Mariana Noreña Vásquez}
            
        \vfill
            
        \vspace{0.8cm}
            
        \Large
        Despartamento de Ingeniería Electrónica y Telecomunicaciones\\
        Universidad de Antioquia\\
        Medellín\\
        Marzo de 2021
            
    \end{center}
\end{titlepage}

\tableofcontents
\newpage
\section{Sección introductoria}\label{intro}
En el presente se llevará a cabo una secuencia de instrucciones para el desarrollo de la actividad teniendo en cuenta los siguientes elementos: una hoja de papel y dos tarjetas.

\section{Sección de contenido} \label{contenido}
A continuación se explicará, paso a paso, las instrucciones que se deben seguir para el desarrollo de esta actividad. Cabe aclarar que la misma se hará únicamente con la mano dominante.
\begin{enumerate}
    \item Se parte de la posición inicial establecida.
    \item Tome la hoja del extremo inferior izquierdo con los dedos pulgar e índice de la mano dominante. 
    \item Voltee la mano hacia la derecha hasta que toque la mesa y suelte la hoja.
    \item Conservando la posición que tienen las tarjetas y sin llegar a separarlas, ponga el dedo índice en el lado derecho de las tarjetas y el dedo pulgar en el lado izquierdo de las mismas.
    \item Posteriormente, levántelas de su posición y ubíquelas  encima de la hoja de tal manera que queden dentro de ella.
    \item Se procede a poner el dedo pulgar en el lado izquierdo y el dedo del medio en el lado derecho de las tarjetas. 
    \item Levante las antes mencionadas, ubicándolas verticalmente. 
    \item Posicione el dedo índice en el limite de la parte superior de las tarjetas. 
    \item Aún estando en la posición vertical, baje su mano hasta que ellas toquen la hoja.
    \item Deje el dedo índice sosteniendo las tarjetas, y aleje los otros dos dedos, con que las sostiene, de ellas. 
    \item Con el dedo pulgar y del medio separe la tarjeta que apunta hacia su mano, sin dejar de hacer presión con el dedo índice, de tal forma que las tarjetas y la base de la hoja formen un triángulo isósceles.
    \item Por último, retire los 3 dedos dedos de la tarjeta.
\end{enumerate}

\section{Sección de conclusión} \label{contenido}
Al finalizar la actividad, los 3 participantes coincidieron en que la parte más complicada fue llevar a cabo las instrucciones planteadas, en la sección anterior, solamente con su mano dominante en vez de tener la posibilidad de usar ambas manos y más aún emplear solo algunos dedos de la antes mencionada.
Por otro lado, fue todo un reto plantear de forma adecuada la secuencialidad de las instrucciones y que, de igual manera, fuera entendible y claro cada paso que se debía realizar. Sin embargo, se debe de mejorar en la especificación y puntualidad de los pasos a dar para así evitar la ambigüedad.


\end{document}
